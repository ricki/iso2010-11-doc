\documentclass[a4paper,11pt,oneside]{report}
\usepackage[pdftex]{graphicx}
\usepackage[utf8x]{inputenc}
\usepackage[spanish]{babel}
\usepackage{fancyhdr}
\usepackage[Bjornstrup]{fncychap}
\usepackage{tabularx}
\usepackage{hyperref}
\usepackage{eurosym}
\usepackage{longtable}
% \usepackage[usenames,dvipsnames]{color}
% \usepackage{colortbl}
% \usepackage[caption=false]{subfig}
% \usepackage{float}
% \usepackage{pdflscape}

\setlength{\headheight}{25pt}
\setlength{\parskip}{6pt}

% Margenes 1cm mas pequennos
% \addtolength{\oddsidemargin}{-1cm}
% \addtolength{\evensidemargin}{-1cm}
% \addtolength{\textwidth}{2cm}
% \addtolength{\voffset}{-1cm}
% \addtolength{\textheight}{2cm}

\lhead{\includegraphics[height=20pt]{logo-umbrella.png}}
\chead{}
\rhead{\nouppercase{\leftmark}}

\hypersetup{
colorlinks,
citecolor=black,
filecolor=black,
linkcolor=black,
urlcolor=black
}

 % La que hay que liar para que las cabeceras de los capitulos
 % no tiren media pagina a la basura...

\makeatletter
\def\@makechapterhead#1{{
\parindent \z@ \raggedright \normalfont
\ifnum \c@secnumdepth >\m@ne
	\if@mainmatter
		\DOCH
	\fi
\fi
\interlinepenalty\@M
\if@mainmatter
	\DOTI{#1}
\else
	\DOTIS{#1}
\fi
}}

\def\@makeschapterhead#1{{
\parindent \z@ \raggedright
\normalfont
\interlinepenalty\@M
\DOTIS{#1}
}}
\makeatother

\begin{document}

\renewcommand\listtablename{Índice de tablas}
\renewcommand\tablename{Tabla}

\pagestyle{plain}

%%%% Title Page %%%%%

\pagenumbering{alph}

\begin{titlepage}
\begin{center}

% Logo
\includegraphics[width=0.6\textwidth]{logo-umbrella.png}\\[4cm]

% Title
{\huge \textbf{La Conquista del Mundo}}\\[0.5cm]
{\huge {Plan de gestión de proyecto software}}\\[0.5cm]
{\Large \textbf{v1.0}}\\[4cm]

% Authors
\begin{minipage}{0.5\textwidth}
\large
\hspace{1cm}\textbf{\emph{Equipo de desarrollo}}\\
Ángel Durán Izquierdo\\
Antonio Gómez Poblete\\
Antonio Martín Menor de Santos\\
Daniel León Romero\\
Jorge Colao Adán\\
Laura Núñez Villa\\
Ricardo Ruedas García\\
\end{minipage}\\[2cm]

{\Large \today}
\end{center}
\end{titlepage}

%%%% end Title Page %%%%%

\clearpage
\pagenumbering{arabic}

\chapter*{Historial de cambios}
\addcontentsline{toc}{chapter}{Historial de cambios}
\setcounter{page}{2}

\begin{center}
\begin{tabularx}{\textwidth}{p{0.1\textwidth}p{0.2\textwidth}X}
\textbf{v1.0} & 09-10-2010 & Versión incial del plan de gestión de proyecto
software.
\end{tabularx}
\end{center}

\clearpage

\chapter*{Prefacio}
\addcontentsline{toc}{chapter}{Prefacio}

Este documento pretende especificar el plan de proyecto para desarrollar la
primera práctica de la asignatura de Ingeniería del Software II, de nombre
\textit{La Conquista del Mundo}, durante el curso 2010/11 en la Escuela Superior
de Informática de la Universidad de Castilla -- La Mancha.

Este plan está dirigido a los profesores de la asignatura, Ismael Caballero
Muñoz-Reja y Macario Polo Usaola, así como al resto resto de integrantes del
grupo de prácticas, Ángel Durán Izquierdo, Antonio Gómez Poblete, Antonio Martín
Menor de Santos, Daniel León Romero, Jorge Colao Adán y Laura Núñez Villa.

Dada durante las clases de la asignatura la sugerencia de elegir un nombre para
el grupo de prácticas, los integrantes de este grupo han decidido tomar el
nombre de \textit{Umbrella Corporation}, empresa ficticia que aparece en un
conocido videojuego. Este grupo tomará ese nombre con su correspondiente logo
como sus señas de identidad, y éstos aparecerán en toda la documentación que
sea necesario generar.

\clearpage

\tableofcontents
\addcontentsline{toc}{chapter}{\contentsname}
% \listoffigures
% \listoftables

\clearpage

\pagestyle{fancy}

\chapter{Visión general}
\section{Resumen del proyecto}

\subsection{Propósito, alcance y objetivos}

\textit{La Conquista del Mundo} es un juego de estrategia en tiempo real que es
jugado sobre un tablero con regiones que representan el mundo. Este juego sigue
una arquitectura cliente-servidor y hace uso de un motor de comunicaciones de
red para conseguirlo.

El propósito de este proyecto es pues realizar todo el ciclo de desarrollo hasta
conseguir implementar un cliente para ordenador de \textit{La Conquista del
Mundo}.

Se consideran dentro del alcance del proyecto no sólo las actividades de
desarrollo del producto, sino también las relacionadas con conseguir un entorno
de desarrollo y ejecución adecuados.

El objetivo principal del proyecto será la realización de la práctica de manera
satisfactoria antes de la fecha de entrega. Además habrá otros objetivos
secundarios, en su mayor parte formativos, como el conocimiento y puesta en
práctica del Proceso Unificado de Desarrollo o el uso de repositorios software
en proyectos con varios integrantes.

\subsection{Suposiciones y restricciones}

Para el desarrollo de este proyecto no será necesario ningún tipo de hardware
especial; los ordenadores personales de los propios desarrolladores serán
suficiente.

Las herramientas software necesarias estarán a disposición de los
desarrolladores a coste cero, bien porque sean herramientas libres o porque la
universidad proporcione las licencias correspondientes.

Se entiende que los desarrolladores son estudiantes, y que por tanto el tiempo
que podrán dedicar al proyecto es limitado. Este hecho debe estar reflejado en
la carga semanal de trabajo.

Por el caracter formativo del proyecto, cada desarrollador deberá adoptar al
menos una vez durante el desarrollo del proyecto cada uno de los siguientes
roles: analista, diseñador, programador y probador.

Este proyecto depende del proyecto encargado del servidor de la aplicación. Las
comunicaciones con el servidor se realizarán mediante la tecnología RMI y la
interfaz de comunicaciones será definida por el equipo del servidor.

Se deberá entregar a los destinatarios de este proyecto una máquina virtual
para VirtualBox que contendrá todo el desarrollo del proyecto así como un
entorno donde ejecutar la aplicación. El sistema operativo de esa máquina
virtual queda a elección del grupo.

El número de componentes del equipo de desarrollo estaba limitado a un máximo
de ocho personas, quedando el equipo finalmente formado por siete personas.

El tiempo para desarrollar la aplicación vence el día 31 de enero de 2011.

\subsection{Entregables del proyecto}

El producto final del desarrollo será una máquina virtual para VirtualBox. Será
entregado en formato DVD-R y contendrá los siguientes elementos:
\begin{itemize}
\item La aplicación final y todas las dependencias necesarias para su correcta
ejecución.
\item Manual de usuario.
\item El proyecto de desarrollo con el código fuente completo de la aplicación.
\item La documentación del proyecto compuesta por:
\subitem Plan de gestión de proyecto software
\subitem Especificación de requisitos software
\subitem Diagramas de análisis y diseño de clases
\subitem Plan de pruebas software
\end{itemize}

\subsection{Resumen de la planificación y el presupuesto}

\section{Evolución del plan}


\chapter{Referencias}
El documento que ha dado inicio al proyecto se puede consultar en la siguiente
dirección:\\
\\
{\small
\url{https://campusvirtual.uclm.es/file.php/11303/PRG1P/PrISO2-1Pv1.1.pdf
}}\\

Para la elaboración de este documento se ha seguido ampliamente el siguiente
estándar:\\
\\
IEEE Std 1058-1998, \textit{Standard for Software Project Management Plans}.
The Institute of Electrical and Electronics Engineers, Inc. 1998.\\

Además, se han tomado como ejemplo a seguir los siguientes documentos:\\
\\
{\small
\url{http://www.buckley-golder.com/papers/mbg_SPMP_ProjectManagement.pdf}\\
\url{http://sourcefrog.net/projects/newuserfs/doc/plan/plan.pdf}\\
}\\


\chapter{Definiciones}
\begin{description}
\item [GIT] Sistema de control de versiones distribuido.
\item [RMI] Remote Method Invocation. Interfaz de comunicaciones de red para el
lenguaje de programación Java.
\end{description}



\chapter{Organización del proyecto}
\section{Interfaces externas}

Los profesores de la asignatura harán las partes de clientes y serán los que, al
final del desarrollo, evaluen el proyecto. Es pues a ellos a quien se deben
dirigir las dudas o problemas que surgan durante el desarrollo.

Paralelamente, otro grupo de la asignatura se encargará del desarrollo de la
parte servidor de la aplicación. Los grupos de trabajo encargados de los
clientes, entre los que se incluye el nuestro, deberán ponerse de acuerdo con
ellos y establecer una interfaz de comunicaciones común para todos.

\section{Estructura interna}

El equipo de desarrollo está formado por siete alumnos. Uno de ellos se
encargará de las labores de gestión y organización en sus funciones de director
de proyecto. El resto del equipo se encargará del desarrollo de la aplicación y
trabajará bien de manera individual o en pequeños grupos que variarán de manera
periódica.

\section{Roles y responsabilidades}

Este proyecto dispone de los siguientes roles y responsabilidades:
\begin{itemize}
\item \textbf{Director de proyecto:} Está encargado de la gestión,
planificación y documentación del proyecto. Además deberá proveer al resto del
equipo con las herramientas adecuadas para el desarrollo.
\item \textbf{Analista:} Participará en la captura de requisitos y en el
análisis de casos de uso.
\item \textbf{Diseñador:} Participará en el diseño de la aplicación y en la
preparación de los planes de prueba.
\end{itemize}



\chapter{Planes de procesos de gestión}
\section{Plan de inicio del proyecto}

\subsection{Plan de estimación}

Para la estimación del proyecto, éste se ha estructurado en una serie de tareas
y se ha estimado un esfuerzo para cada tarea basado en la experiencia personal
del equipo de desarrollo en proyectos de características similares. Ese esfuerzo
estará expresado en horas y representará el tiempo necesario por una persona
para realizar dicha tarea.

Para la planificación del calendario del proyecto se ha tenido en cuenta un
esfuerzo de una hora diaria de media a la semana por cada integrante del grupo,
esto es, un total de 5 horas semanales.

\subsection{Plan de personal}

El grupo está integrado por un jefe de proyecto y seis desarrolladores. Éstos
forman parte del grupo desde su formación al inicio de la asignatura y
permanecerán en él durante todo el curso.

\subsection{Plan de adquisición de recursos}

En el desarrollo del proyecto se hará uso de dos aplicaciones fundamentales:
\href{http://www.eclipse.org}{Eclipse} y
\href{http://www.visual-paradigm.com}{Visual Paradigm}.
Ambas aplicaciones se pueden obtener libremente de sus respectivos sitios web,
y sólo en el caso de Visual Paradigm hará falta una licencia, que será
proporcionada por los profesores de la asignatura.

Se creará una lista de correo en \href{http://groups.google.com}{Google Groups}
para las comunicaciones internas del grupo. Todos los integrantes del grupo,
así como los profesores de la asignatura, recibirán una invitación a la lista
tan pronto como sea creada.

El trabajo de desarrollo se llevará a cabo en dos repositorios Git, uno dedicado
a la documentación del proyecto y otro al proyecto en sí. Ambos estarán
hospedados en \href{http://github.com}{GitHub}, donde deberán registrarse todos
los integrantes del grupo.

\subsection{Plan de formación de personal}

Los integrantes del grupo recibirán por parte de los profesores de la
asignatura a lo largo del curso los conocimientos y la ayuda necesaria para
llevar a cabo este proyecto.

\section{Plan de trabajo}

\subsection{Actividades}

En la siguiente estructura de descomposición de trabajo se muestran las tareas
que componen el proyecto así como el esfuerzo estimado para su realización.

{\footnotesize
\begin{longtable}[c]{lrl}
\caption{Estructura de descomposición de trabajo} \\

\textbf{Tarea}&\textbf{Esfuerzo} & \textbf{Descripción} \\
\hline \hline
\endhead

1 & & \textbf{Inicio del proyecto} \\
1.1 & 30 h & Elaboración del plan de gestión de proyecto software. \\
1.2 & 30 h & Elaboración de la especificación de requisitos software. \\
    & 60 h & \textbf{Total} \\
\hline

2 & & \textbf{Análisis y diseño} \\
2.1  & 2 h & Caso de uso: Registrarse. \\
2.2  & 3 h & Caso de uso: Iniciar sesión. \\
2.3  & 2 h & Caso de uso: Cerrar sesión. \\
2.4  & 2 h & Caso de uso: Crear una nueva partida. \\
2.5  & 2 h & Caso de uso: Unirse a una partida. \\
2.6  & 4 h & Caso de uso: Conectarse a una partida. \\
2.7  & 2 h & Caso de uso: Desconectarse de una partida. \\
2.8  & 3 h & Caso de uso: Ver la lista de partidas. \\
2.9  & 2 h & Caso de uso: Recibir notificación de un nuevo jugador. \\
2.10 & 5 h & Caso de uso: Realizar un movimiento. \\
2.11 & 4 h & Caso de uso: Responder a un movimiento. \\
2.12 & 4 h & Caso de uso: Comprar refuerzos. \\
2.13 & 2 h & Caso de uso: Enviar petición de alianza. \\
2.14 & 2 h & Caso de uso: Recibir petición de alianza. \\
2.15 & 2 h & Caso de uso: Romper alianza. \\
2.16 & 2 h & Caso de uso: Enviar actualización de la partida. \\
2.17 & 2 h & Caso de uso: Recibir actualización de la partida. \\
    & 45 h & \textbf{Total} \\
\hline

3 & & \textbf{Implementación} \\
3.1  &  5 h & Caso de uso: Registrarse. \\
3.2  &  5 h & Caso de uso: Iniciar sesión. \\
3.3  &  2 h & Caso de uso: Cerrar sesión. \\
3.4  &  8 h & Caso de uso: Crear una nueva partida. \\
3.5  &  4 h & Caso de uso: Unirse a una partida. \\
3.6  & 25 h & Caso de uso: Conectarse a una partida. \\
3.7  &  2 h & Caso de uso: Desconectarse de una partida. \\
3.8  & 10 h & Caso de uso: Ver la lista de partidas. \\
3.9  &  2 h & Caso de uso: Recibir notificación de un nuevo jugador. \\
3.10 & 25 h & Caso de uso: Realizar un movimiento. \\
3.11 & 10 h & Caso de uso: Responder a un movimiento. \\
3.12 &  6 h & Caso de uso: Comprar refuerzos. \\
3.13 &  4 h & Caso de uso: Enviar petición de alianza. \\
3.14 &  4 h & Caso de uso: Recibir petición de alianza. \\
3.15 &  4 h & Caso de uso: Romper alianza. \\
3.16 &  4 h & Caso de uso: Enviar actualización de la partida. \\
3.17 &  4 h & Caso de uso: Recibir actualización de la partida. \\
    & 124 h & \textbf{Total} \\
\hline

4 & & \textbf{Pruebas} \\
4.1 & 10 h & Implementación del servidor de prueba. \\
4.2 & 20 h & Pruebas unitarias. \\
4.3 & 20 h & Pruebas de integración. \\
4.4 & 20 h & Pruebas de sistema. \\
    & 70 h & \textbf{Total} \\
\hline
   & 299 h & \textbf{Total global} \\
\hline
\end{longtable}
}

\subsection{Planificación}

A continuación se muestra el calendario que se seguirá para el desarrollo del
proyecto. Cada iteración se corresponde con una semana de acuerdo a la
siguiente lista de fechas:

\begin{itemize}
\item \textbf{Fase de inicio}
\subitem Iteración 1: del 4 al 8 de octubre.
\subitem Iteración 2: del 11 al 15 de octubre.
\subitem Iteración 3: del 18 al 22 de octubre.
\item \textbf{Fase de elaboración}
\subitem Iteración 4: del 25 al 29 de octubre.
\subitem Iteración 5: del 1 al 5 de noviembre.
\subitem Iteración 6: del 8 al 12 de noviembre.
\item \textbf{Fase de construcción}
\subitem Iteración 7: del 15 al 19 de noviembre.
\subitem Iteración 8: del 22 al 26 de noviembre.
\subitem Iteración 9: del 29 de noviembre al 3 de diciembre.
\item \textbf{Fase de transición}
\subitem Iteración 10: del 6 al 10 de diciembre.
\subitem Iteración 11: del 13 al 17 de diciembre.
\end{itemize}


{\footnotesize
\begin{longtable}[c]{l|rrr|rrr|rrr|rr|}
\caption{Planificación del proyecto} \\

\textbf{Tarea} & \textbf{It01} & \textbf{It02} &
\textbf{It03} & \textbf{It04} & \textbf{It05} & \textbf{It06} & \textbf{It07} &
\textbf{It08} & \textbf{It09} & \textbf{It10} & \textbf{It11} \\
\hline \hline
\endhead

1.1 & \cellcolor[gray]{0.25} & \cellcolor[gray]{0.25} & \cellcolor[gray]{0.25} &
& & & & & & & \\

1.2 & & \cellcolor[gray]{0.25} & \cellcolor[gray]{0.25} & & & & & & & & \\
\hline

2.1 & & & \cellcolor[gray]{0.25} & & & & & & & & \\
2.2 & & & \cellcolor[gray]{0.25} & & & & & & & & \\
2.3 & & & \cellcolor[gray]{0.25} & & & & & & & & \\
2.4 & & & & \cellcolor[gray]{0.25} & & & & & & & \\
2.5 & & & & \cellcolor[gray]{0.25} & & & & & & & \\
2.6 & & & & \cellcolor[gray]{0.25} & & & & & & & \\
2.7 & & & & \cellcolor[gray]{0.25} & & & & & & & \\
2.8 & & & & \cellcolor[gray]{0.25} & & & & & & & \\
2.9 & & & & \cellcolor[gray]{0.25} & & & & & & & \\
2.10 & & & & & \cellcolor[gray]{0.25} & & & & & & \\
2.11 & & & & & \cellcolor[gray]{0.25} & & & & & & \\
2.12 & & & & & \cellcolor[gray]{0.25} & & & & & & \\
2.13 & & & & & & \cellcolor[gray]{0.25} & & & & & \\
2.14 & & & & & & \cellcolor[gray]{0.25} & & & & & \\
2.15 & & & & & & \cellcolor[gray]{0.25} & & & & & \\
2.16 & & & & & \cellcolor[gray]{0.25} & & & & & & \\
2.17 & & & & & \cellcolor[gray]{0.25} & & & & & & \\
\hline

3.1 & & & & \cellcolor[gray]{0.25} & & & & & & & \\
3.2 & & & & \cellcolor[gray]{0.25} & & & & & & & \\
3.3 & & & & & & \cellcolor[gray]{0.25} & & & & & \\
3.4 & & & & & \cellcolor[gray]{0.25} & & & & & & \\
3.5 & & & & & & \cellcolor[gray]{0.25} & & & & & \\
3.6 & & & & & & \cellcolor[gray]{0.25} & \cellcolor[gray]{0.25} & & & & \\
3.7 & & & & & & & & & & \cellcolor[gray]{0.25} & \\
3.8 & & & & & \cellcolor[gray]{0.25} & & & & & & \\
3.9 & & & & & & \cellcolor[gray]{0.25} & & & & & \\
3.10 & & & & & & & & \cellcolor[gray]{0.25} & & & \\
3.11 & & & & & & & & & \cellcolor[gray]{0.25} & & \\
3.12 & & & & & & & \cellcolor[gray]{0.25} & & & & \\
3.13 & & & & & & & & & & \cellcolor[gray]{0.25} & \\
3.14 & & & & & & & & & & \cellcolor[gray]{0.25} & \\
3.15 & & & & & & & & & & \cellcolor[gray]{0.25} & \\
3.16 & & & & & & & & & \cellcolor[gray]{0.25} & & \\
3.17 & & & & & & & & & \cellcolor[gray]{0.25} & & \\
\hline

4.1 & & & & & & \cellcolor[gray]{0.25} & & & & & \\
4.2 & & & & & & & \cellcolor[gray]{0.25} & \cellcolor[gray]{0.25} & & & \\
4.3 & & & & & & & & \cellcolor[gray]{0.25} & \cellcolor[gray]{0.25} & & \\
4.4 & & & & & & & & & \cellcolor[gray]{0.25} & \cellcolor[gray]{0.25} &
\cellcolor[gray]{0.25} \\
\hline
\end{longtable}
}

\subsection{Asignación de recursos}

Los desarrolladores serán asignados un rol en cada iteración para realizar las
tareas que correspondan. Los distintos roles que podrán tomar son los
siguientes:
\begin{itemize}
\item Recurso 1: Especificador de casos de uso.
\item Recurso 2: Analista-diseñador.
\item Recurso 3: Programador.
\item Recurso 4: Tester.
\end{itemize}


A continuación se muestra la cantidad de recursos
asignados en cada iteración.

{\footnotesize
\begin{longtable}[c]{l|c|c|c|c|}
\caption{Asignación de recursos} \\

 & \textbf{Recurso 1} & \textbf{Recurso 2} & \textbf{Recurso 3}
& \textbf{Recurso 4} \\
\hline \hline
\endhead

Iteración  1 &   &   &   &   \\
Iteración  2 & 4 &   &   &   \\
Iteración  3 & 4 & 2 &   &   \\
Iteración  4 &   & 3 & 2 &   \\
Iteración  5 &   & 3 & 3 &   \\
Iteración  6 &   & 2 & 3 & 1 \\
Iteración  7 &   &   & 4 & 2 \\
Iteración  8 &   &   & 4 & 2 \\
Iteración  9 &   &   & 4 & 2 \\
Iteración 10 &   &   & 2 & 4 \\
Iteración 11 &   &   &   & 4 \\
\hline
\end{longtable}
}

\section{Plan de gestión de riesgos}

El desarrollo del proyecto puede verse afectado por diversos riesgos.

\subsection{Riesgos técnicos}

Algunos de los desarrolladores pueden no haber usado hasta ahora ningún sistema
de control de versiones. Además se hará uso de Git, que es un sistema
bastante reciente y con más funcionalidades que sistemas anteriores como
Subversion, lo que implica también mayor complejidad inicial. Se deberá proveer
a los desarrolladores con la suficiente documentación para un correcto uso del
sistema de control de versiones.

El equipo de desarrollo se enfrenta al diseño de una interfaz gráfica de
usuario. En esta proyecto se pretende aprender buenas prácticas de diseño pero
es difícil seguir estas prácticas sin experiencia previasuficiente. Por eso es
importante la creación de múltiples prototipos y ejemplos mínimos de código que
realimenten a los analistas y diseñadores.

\subsection{Riesgos de recursos}

Es inevitable tener en cuenta que los desarrolladores no pueden dedicarse al
proyecto a tiempo completo debido a que cursen otras asignaturas. Ésto
provocará que en determinadas semanas se reduzca el tiempo disponible para el
proyecto. Se ha realizado una planificación optimista del proyecto con una fecha
de finalización hasta un mes anterior a la fecha de entrega final, de manera
que se pequeños retrasos en las tareas no impliquen un impacto importante.

\subsection{Riesgos externos}

El proyecto no se realiza en solitario sino que interacciona con otros grupos de
trabajo. Esta interacción debe formalizarse con un conjunto de funciones bien
definidas entre los clientes y el servidor. Un retraso en la redacción de estas
funciones puede provocar cambios internos en algunos componentes de la
aplicación.


\chapter{Planes de procesos técnicos}
\section{Modelo de procesos}

El proyecto sigue el proceso unificado de desarrollo, en el que se divide el
proyecto en varias iteraciones en las que se refinan distintos elementos del
proyecto.

El primer documento a desarrollar será la especificación de requisitos
software, donde se plasmarán las funcionalidades que la aplicación debe cubrir.
Este documento incluirá el modelo de casos de uso, que guiará el desarrollo de
la aplicación en las sucesivas iteraciones.

Seguidamente los analistas-diseñadores desarrollarán cada caso de uso,
describiendo su comportamiento mediante diagramas de secuencia y completando con
cada nuevo caso de uso el diagrama de clases de diseño.

Los diagramas de análisis y diseño servirán a los programadores para realizar
la implementación de la aplicación y a los testers para documentar los planes
de pruebas.

\section{Métodos, herramientas y técnicas}

Para la redacción de la documentación del proyecto se usará el lenguaje de
maquetación de documentos \LaTeX, y se recomendará (pero no obligará) el uso
del programa \href{http://kile.sourceforge.net}{Kile} para su edición.

El modelado de la aplicación se realizará con la ayuda de la aplicación
\href{http://www.visual-paradigm.com}{Visual Paradigm}.

La aplicación estará escrita en el lenguaje de programación Java. Deberá tener
interfaz gráfica de usuario, para lo que se usará la librería gráfica Swing.
Además deberá disponer de una capa de comunicaciones con el servidor, para lo
que se utilizará la tecnología RMI. Tanto Swing como RMI vienen incluídas en la
librería estándar de Java. El desarrollo de la aplicación se realizará usando
el IDE \href{http://www.eclipse.org}{Eclipse}.

Por último, todos los documentos serán gestionados por el sistema de control de
versiones \href{http://git-scm.com}{Git}.

\section{Plan de infraestructura}

El desarrollo de la práctica se realizará usando los ordenadores personales de
los propios desarrolladores ya que no existen requisitos de alto rendimiento o
disponibilidad.

Estos equipos podrán disponer del sistema operativo que su usuario estime
oportuno, siempre que sean capaces de ejecutar sin problemas las herramientas
descritas en la sección anterior.

\section{Plan de aceptación del producto}

En las últimas semanas del desarrollo, el esfuerzo se centrará en la
realización de pruebas de sistema y aceptación para asegurarse de que la
aplicación realizada cumple con los requisitos descritos en el enunciado del
problema, y matizados en las diferentes reuniones con los profesores de la
asignatura.


\chapter{Planes de procesos de soporte}
\section{Plan de gestión de la configuración}

La gestión de la configuración del proyecto se realizará usando el
sistema de control de versiones Git, y se usarán los servicios gratuitos de
\href{http://github.com}{GitHub} para hospedar los repositorios, donde cada
integrante del equipo tendrá una cuenta. La relación de cuentas se puede ver en
la siguiente tabla.

\begin{longtable}[c]{ll}
\caption{Relación de cuentas en GitHub} \\
\textbf{Desarrollador} & \textbf{Usuario} \\
\hline \hline
\endhead
Ricardo Ruedas García (Jefe de proyecto) & \texttt{ricki} \\
\hline
Ángel Durán Izquierdo & \texttt{Aduran} \\
Antonio Gómez Poblete & \texttt{pobleteag} \\
Antonio Martín Menor de Santos & \texttt{deejaytoni} \\
Daniel León Romero & \texttt{DaniLR} \\
Jorge Colao Adán & \texttt{JorgeCA} \\
Laura Núñez Villa & \texttt{LauraN} \\
\hline
\end{longtable}

Se utilizará un flujo de trabajo distribuído, en el que cada integrante del
grupo trabajará sobre su repositorio personal. La labor del jefe de proyecto
será la de integrar todos los flujos de trabajo en su repositorio, que será
siempre el repositorio de referencia.


\chapter{Planes adicionales}
\input{08-proc-extra.tex}

\end{document}
