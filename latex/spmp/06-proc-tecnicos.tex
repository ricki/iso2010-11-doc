\section{Modelo de procesos}

El proyecto sigue el proceso unificado de desarrollo, en el que se divide el
proyecto en varias iteraciones en las que se refinan distintos elementos del
proyecto.

El primer documento a desarrollar será la especificación de requisitos
software, donde se plasmarán las funcionalidades que la aplicación debe cubrir.
Este documento incluirá el modelo de casos de uso, que guiará el desarrollo de
la aplicación en las sucesivas iteraciones.

Seguidamente los analistas-diseñadores desarrollarán cada caso de uso,
describiendo su comportamiento mediante diagramas de secuencia y completando con
cada nuevo caso de uso el diagrama de clases de diseño.

Los diagramas de análisis y diseño servirán a los programadores para realizar
la implementación de la aplicación y a los testers para documentar los planes
de pruebas.

\section{Métodos, herramientas y técnicas}

Para la redacción de la documentación del proyecto se usará el lenguaje de
maquetación de documentos \LaTeX, y se recomendará (pero no obligará) el uso
del programa \href{http://kile.sourceforge.net}{Kile} para su edición.

El modelado de la aplicación se realizará con la ayuda de la aplicación
\href{http://www.visual-paradigm.com}{Visual Paradigm}.

La aplicación estará escrita en el lenguaje de programación Java. Deberá tener
interfaz gráfica de usuario, para lo que se usará la librería gráfica Swing.
Además deberá disponer de una capa de comunicaciones con el servidor, para lo
que se utilizará la tecnología RMI. Tanto Swing como RMI vienen incluídas en la
librería estándar de Java. El desarrollo de la aplicación se realizará usando
el IDE \href{http://www.eclipse.org}{Eclipse}.

Por último, todos los documentos serán gestionados por el sistema de control de
versiones \href{http://git-scm.com}{Git}.

\section{Plan de infraestructura}

El desarrollo de la práctica se realizará usando los ordenadores personales de
los propios desarrolladores ya que no existen requisitos de alto rendimiento o
disponibilidad.

Estos equipos podrán disponer del sistema operativo que su usuario estime
oportuno, siempre que sean capaces de ejecutar sin problemas las herramientas
descritas en la sección anterior.

\section{Plan de aceptación del producto}

En las últimas semanas del desarrollo, el esfuerzo se centrará en la
realización de pruebas de sistema y aceptación para asegurarse de que la
aplicación realizada cumple con los requisitos descritos en el enunciado del
problema, y matizados en las diferentes reuniones con los profesores de la
asignatura.
