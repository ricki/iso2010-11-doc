\section{Plan de inicio del proyecto}

\subsection{Plan de estimación}

Para la estimación del proyecto, éste se ha estructurado en una serie de tareas
y se ha estimado un esfuerzo para cada tarea basado en la experiencia personal
del equipo de desarrollo en proyectos de características similares. Ese esfuerzo
estará expresado en horas y representará el tiempo necesario por una persona
para realizar dicha tarea.

Para la planificación del calendario del proyecto se ha tenido en cuenta un
esfuerzo de una hora diaria de media a la semana por cada integrante del grupo,
esto es, un total de 5 horas semanales.

\subsection{Plan de personal}

El grupo está integrado por un jefe de proyecto y seis desarrolladores. Éstos
forman parte del grupo desde su formación al inicio de la asignatura y
permanecerán en él durante todo el curso.

\subsection{Plan de adquisición de recursos}

En el desarrollo del proyecto se hará uso de dos aplicaciones fundamentales:
\href{http://www.eclipse.org}{Eclipse} y
\href{http://www.visual-paradigm.com}{Visual Paradigm}.
Ambas aplicaciones se pueden obtener libremente de sus respectivos sitios web,
y sólo en el caso de Visual Paradigm hará falta una licencia, que será
proporcionada por los profesores de la asignatura.

Se creará una lista de correo en \href{http://groups.google.com}{Google Groups}
para las comunicaciones internas del grupo. Todos los integrantes del grupo,
así como los profesores de la asignatura, recibirán una invitación a la lista
tan pronto como sea creada.

El trabajo de desarrollo se llevará a cabo en dos repositorios Git, uno dedicado
a la documentación del proyecto y otro al proyecto en sí. Ambos estarán
hospedados en \href{http://github.com}{GitHub}, donde deberán registrarse todos
los integrantes del grupo.

\subsection{Plan de formación de personal}

Los integrantes del grupo recibirán por parte de los profesores de la
asignatura a lo largo del curso los conocimientos y la ayuda necesaria para
llevar a cabo este proyecto.

\section{Plan de trabajo}

\subsection{Actividades}

En la siguiente estructura de descomposición de trabajo se muestran las tareas
que componen el proyecto así como el esfuerzo estimado para su realización.

{\footnotesize
\begin{longtable}[c]{lrl}
\caption{Estructura de descomposición de trabajo} \\

\textbf{Tarea}&\textbf{Esfuerzo} & \textbf{Descripción} \\
\hline \hline
\endhead

1 & & \textbf{Inicio del proyecto} \\
1.1 & 30 h & Elaboración del plan de gestión de proyecto software. \\
1.2 & 30 h & Elaboración de la especificación de requisitos software. \\
    & 60 h & \textbf{Total} \\
\hline

2 & & \textbf{Análisis y diseño} \\
2.1  & 2 h & Caso de uso: Registrarse. \\
2.2  & 3 h & Caso de uso: Iniciar sesión. \\
2.3  & 2 h & Caso de uso: Cerrar sesión. \\
2.4  & 2 h & Caso de uso: Crear una nueva partida. \\
2.5  & 2 h & Caso de uso: Unirse a una partida. \\
2.6  & 4 h & Caso de uso: Conectarse a una partida. \\
2.7  & 2 h & Caso de uso: Desconectarse de una partida. \\
2.8  & 3 h & Caso de uso: Ver la lista de partidas. \\
2.9  & 2 h & Caso de uso: Recibir notificación de un nuevo jugador. \\
2.10 & 5 h & Caso de uso: Realizar un movimiento. \\
2.11 & 4 h & Caso de uso: Responder a un movimiento. \\
2.12 & 4 h & Caso de uso: Comprar refuerzos. \\
2.13 & 2 h & Caso de uso: Enviar petición de alianza. \\
2.14 & 2 h & Caso de uso: Recibir petición de alianza. \\
2.15 & 2 h & Caso de uso: Romper alianza. \\
2.16 & 2 h & Caso de uso: Enviar actualización de la partida. \\
2.17 & 2 h & Caso de uso: Recibir actualización de la partida. \\
    & 45 h & \textbf{Total} \\
\hline

3 & & \textbf{Implementación} \\
3.1  &  5 h & Caso de uso: Registrarse. \\
3.2  &  5 h & Caso de uso: Iniciar sesión. \\
3.3  &  2 h & Caso de uso: Cerrar sesión. \\
3.4  &  8 h & Caso de uso: Crear una nueva partida. \\
3.5  &  4 h & Caso de uso: Unirse a una partida. \\
3.6  & 25 h & Caso de uso: Conectarse a una partida. \\
3.7  &  2 h & Caso de uso: Desconectarse de una partida. \\
3.8  & 10 h & Caso de uso: Ver la lista de partidas. \\
3.9  &  2 h & Caso de uso: Recibir notificación de un nuevo jugador. \\
3.10 & 25 h & Caso de uso: Realizar un movimiento. \\
3.11 & 10 h & Caso de uso: Responder a un movimiento. \\
3.12 &  6 h & Caso de uso: Comprar refuerzos. \\
3.13 &  4 h & Caso de uso: Enviar petición de alianza. \\
3.14 &  4 h & Caso de uso: Recibir petición de alianza. \\
3.15 &  4 h & Caso de uso: Romper alianza. \\
3.16 &  4 h & Caso de uso: Enviar actualización de la partida. \\
3.17 &  4 h & Caso de uso: Recibir actualización de la partida. \\
    & 124 h & \textbf{Total} \\
\hline

4 & & \textbf{Pruebas} \\
4.1 & 10 h & Implementación del servidor de prueba. \\
4.2 & 20 h & Pruebas unitarias. \\
4.3 & 20 h & Pruebas de integración. \\
4.4 & 20 h & Pruebas de sistema. \\
    & 70 h & \textbf{Total} \\
\hline
   & 299 h & \textbf{Total global} \\
\hline
\end{longtable}
}

\subsection{Planificación}

A continuación se muestra el calendario que se seguirá para el desarrollo del
proyecto. Cada iteración se corresponde con una semana de acuerdo a la
siguiente lista de fechas:

\begin{itemize}
\item \textbf{Fase de inicio}
\subitem Iteración 1: del 4 al 8 de octubre.
\subitem Iteración 2: del 11 al 15 de octubre.
\subitem Iteración 3: del 18 al 22 de octubre.
\item \textbf{Fase de elaboración}
\subitem Iteración 4: del 25 al 29 de octubre.
\subitem Iteración 5: del 1 al 5 de noviembre.
\subitem Iteración 6: del 8 al 12 de noviembre.
\item \textbf{Fase de construcción}
\subitem Iteración 7: del 15 al 19 de noviembre.
\subitem Iteración 8: del 22 al 26 de noviembre.
\subitem Iteración 9: del 29 de noviembre al 3 de diciembre.
\item \textbf{Fase de transición}
\subitem Iteración 10: del 6 al 10 de diciembre.
\subitem Iteración 11: del 13 al 17 de diciembre.
\end{itemize}


{\footnotesize
\begin{longtable}[c]{l|rrr|rrr|rrr|rr|}
\caption{Planificación del proyecto} \\

\textbf{Tarea} & \textbf{It01} & \textbf{It02} &
\textbf{It03} & \textbf{It04} & \textbf{It05} & \textbf{It06} & \textbf{It07} &
\textbf{It08} & \textbf{It09} & \textbf{It10} & \textbf{It11} \\
\hline \hline
\endhead

1.1 & \cellcolor[gray]{0.25} & \cellcolor[gray]{0.25} & \cellcolor[gray]{0.25} &
& & & & & & & \\

1.2 & & \cellcolor[gray]{0.25} & \cellcolor[gray]{0.25} & & & & & & & & \\
\hline

2.1 & & & \cellcolor[gray]{0.25} & & & & & & & & \\
2.2 & & & \cellcolor[gray]{0.25} & & & & & & & & \\
2.3 & & & \cellcolor[gray]{0.25} & & & & & & & & \\
2.4 & & & & \cellcolor[gray]{0.25} & & & & & & & \\
2.5 & & & & \cellcolor[gray]{0.25} & & & & & & & \\
2.6 & & & & \cellcolor[gray]{0.25} & & & & & & & \\
2.7 & & & & \cellcolor[gray]{0.25} & & & & & & & \\
2.8 & & & & \cellcolor[gray]{0.25} & & & & & & & \\
2.9 & & & & \cellcolor[gray]{0.25} & & & & & & & \\
2.10 & & & & & \cellcolor[gray]{0.25} & & & & & & \\
2.11 & & & & & \cellcolor[gray]{0.25} & & & & & & \\
2.12 & & & & & \cellcolor[gray]{0.25} & & & & & & \\
2.13 & & & & & & \cellcolor[gray]{0.25} & & & & & \\
2.14 & & & & & & \cellcolor[gray]{0.25} & & & & & \\
2.15 & & & & & & \cellcolor[gray]{0.25} & & & & & \\
2.16 & & & & & \cellcolor[gray]{0.25} & & & & & & \\
2.17 & & & & & \cellcolor[gray]{0.25} & & & & & & \\
\hline

3.1 & & & & \cellcolor[gray]{0.25} & & & & & & & \\
3.2 & & & & \cellcolor[gray]{0.25} & & & & & & & \\
3.3 & & & & & & \cellcolor[gray]{0.25} & & & & & \\
3.4 & & & & & \cellcolor[gray]{0.25} & & & & & & \\
3.5 & & & & & & \cellcolor[gray]{0.25} & & & & & \\
3.6 & & & & & & \cellcolor[gray]{0.25} & \cellcolor[gray]{0.25} & & & & \\
3.7 & & & & & & & & & & \cellcolor[gray]{0.25} & \\
3.8 & & & & & \cellcolor[gray]{0.25} & & & & & & \\
3.9 & & & & & & \cellcolor[gray]{0.25} & & & & & \\
3.10 & & & & & & & & \cellcolor[gray]{0.25} & & & \\
3.11 & & & & & & & & & \cellcolor[gray]{0.25} & & \\
3.12 & & & & & & & \cellcolor[gray]{0.25} & & & & \\
3.13 & & & & & & & & & & \cellcolor[gray]{0.25} & \\
3.14 & & & & & & & & & & \cellcolor[gray]{0.25} & \\
3.15 & & & & & & & & & & \cellcolor[gray]{0.25} & \\
3.16 & & & & & & & & & \cellcolor[gray]{0.25} & & \\
3.17 & & & & & & & & & \cellcolor[gray]{0.25} & & \\
\hline

4.1 & & & & & & \cellcolor[gray]{0.25} & & & & & \\
4.2 & & & & & & & \cellcolor[gray]{0.25} & \cellcolor[gray]{0.25} & & & \\
4.3 & & & & & & & & \cellcolor[gray]{0.25} & \cellcolor[gray]{0.25} & & \\
4.4 & & & & & & & & & \cellcolor[gray]{0.25} & \cellcolor[gray]{0.25} &
\cellcolor[gray]{0.25} \\
\hline
\end{longtable}
}

\subsection{Asignación de recursos}

Los desarrolladores serán asignados un rol en cada iteración para realizar las
tareas que correspondan. Los distintos roles que podrán tomar son los
siguientes:
\begin{itemize}
\item Recurso 1: Especificador de casos de uso.
\item Recurso 2: Analista-diseñador.
\item Recurso 3: Programador.
\item Recurso 4: Tester.
\end{itemize}


A continuación se muestra la cantidad de recursos
asignados en cada iteración.

{\footnotesize
\begin{longtable}[c]{l|c|c|c|c|}
\caption{Asignación de recursos} \\

 & \textbf{Recurso 1} & \textbf{Recurso 2} & \textbf{Recurso 3}
& \textbf{Recurso 4} \\
\hline \hline
\endhead

Iteración  1 &   &   &   &   \\
Iteración  2 & 4 &   &   &   \\
Iteración  3 & 4 & 2 &   &   \\
Iteración  4 &   & 3 & 2 &   \\
Iteración  5 &   & 3 & 3 &   \\
Iteración  6 &   & 2 & 3 & 1 \\
Iteración  7 &   &   & 4 & 2 \\
Iteración  8 &   &   & 4 & 2 \\
Iteración  9 &   &   & 4 & 2 \\
Iteración 10 &   &   & 2 & 4 \\
Iteración 11 &   &   &   & 4 \\
\hline
\end{longtable}
}

\section{Plan de gestión de riesgos}

El desarrollo del proyecto puede verse afectado por diversos riesgos.

\subsection{Riesgos técnicos}

Algunos de los desarrolladores pueden no haber usado hasta ahora ningún sistema
de control de versiones. Además se hará uso de Git, que es un sistema
bastante reciente y con más funcionalidades que sistemas anteriores como
Subversion, lo que implica también mayor complejidad inicial. Se deberá proveer
a los desarrolladores con la suficiente documentación para un correcto uso del
sistema de control de versiones.

El equipo de desarrollo se enfrenta al diseño de una interfaz gráfica de
usuario. En esta proyecto se pretende aprender buenas prácticas de diseño pero
es difícil seguir estas prácticas sin experiencia previasuficiente. Por eso es
importante la creación de múltiples prototipos y ejemplos mínimos de código que
realimenten a los analistas y diseñadores.

\subsection{Riesgos de recursos}

Es inevitable tener en cuenta que los desarrolladores no pueden dedicarse al
proyecto a tiempo completo debido a que cursen otras asignaturas. Ésto
provocará que en determinadas semanas se reduzca el tiempo disponible para el
proyecto. Se ha realizado una planificación optimista del proyecto con una fecha
de finalización hasta un mes anterior a la fecha de entrega final, de manera
que se pequeños retrasos en las tareas no impliquen un impacto importante.

\subsection{Riesgos externos}

El proyecto no se realiza en solitario sino que interacciona con otros grupos de
trabajo. Esta interacción debe formalizarse con un conjunto de funciones bien
definidas entre los clientes y el servidor. Un retraso en la redacción de estas
funciones puede provocar cambios internos en algunos componentes de la
aplicación.
