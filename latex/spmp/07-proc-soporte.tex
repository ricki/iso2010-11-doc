\section{Plan de gestión de la configuración}

La gestión de la configuración del proyecto se realizará usando el
sistema de control de versiones Git, y se usarán los servicios gratuitos de
\href{http://github.com}{GitHub} para hospedar los repositorios, donde cada
integrante del equipo tendrá una cuenta. La relación de cuentas se puede ver en
la siguiente tabla.

\begin{longtable}[c]{ll}
\caption{Relación de cuentas en GitHub} \\
\textbf{Desarrollador} & \textbf{Usuario} \\
\hline \hline
\endhead
Ricardo Ruedas García (Jefe de proyecto) & \texttt{ricki} \\
\hline
Ángel Durán Izquierdo & \texttt{Aduran} \\
Antonio Gómez Poblete & \texttt{pobleteag} \\
Antonio Martín Menor de Santos & \texttt{deejaytoni} \\
Daniel León Romero & \texttt{DaniLR} \\
Jorge Colao Adán & \texttt{JorgeCA} \\
Laura Núñez Villa & \texttt{LauraN} \\
\hline
\end{longtable}

Se utilizará un flujo de trabajo distribuído, en el que cada integrante del
grupo trabajará sobre su repositorio personal. La labor del jefe de proyecto
será la de integrar todos los flujos de trabajo en su repositorio, que será
siempre el repositorio de referencia para el resto del grupo.

\section{Plan de verificación y validación}

Para las pruebas unitarias, se utilizará la suite de pruebas JUnit disponible
para Java en combinación con la herramienta EclEmma para Eclipse.

Para las pruebas exploratorias será necesario implementar un servidor sin
funcionalidad para poder detectar cualquier problema de la aplicación en un
enterno de ejecución lo más parecido posible al entorno final.

\section{Plan de documentación}

La documentación del proyecto estará siempre localizable en el sistema de
control de versiones. Se puede visitar en formato web en la siguiente
dirección:\\
\\
{\small
\url{http://github.com/ricki/iso2010-11-doc}
}\\

El proyecto realizado con Visual Paradigm se encuentra en el directorio
\texttt{uml}. El resto de documentación se encuentra en el directorio
\texttt{latex}. Adicionalmente se añadirá una versión compilada en PDF de los
documentos listos para distribución en el directorio \texttt{distrib}.

El código para identificar los documentos empezará por las letras UC, siglas de
Umbrella Corporation. Le seguirán un número de dos cifras indicando el tipo de
documento, un guión, y un número de tres cifras para identificar un documento
de una categoría. Después se añadirá un punto y la versión del documento.

Los códigos de categoría usados son los siguientes:

\begin{itemize}
\item $01$ - Documentación de la fase de inicio.
\item $02$ - Actas de reuniones.
\item $03$ - Planes de pruebas.
\end{itemize}

Por ejemplo, el código del presente documento será el siguiente:

\begin{center}
\texttt{UC 01-001.01}
\end{center}

Es decir, la primera versión del primer documento de la fase de inicio.

\section{Plan de garantías de calidad}

Según se aproximen las últimas iteraciones del desarrollo, los recursos
destinados a probar el sistema aumentarán con respecto al resto de flujos de
trabajo. Serán las pruebas exploratorias de la aplicación ya en su entorno de
ejecución las que determinen si la aplicación cumple con los requisitos del
enunciado.

Se espera que para cuando empiecen las pruebas exploratorias, esté ya a
disposición de todos los equipos de trabajo un servidor activo a todas horas
sobre el que probar la aplicación.

\section{Plan de revisiones y auditorías}

Antes de la entrega final, se realizarán varias reuniones de grupo con los
profesores de la asignatura que revisarán el progreso del proyecto e informarán
al grupo sobre los aspectos a corregir.

A día de hoy hay planificadas las siguientes reuniones:

\begin{itemize}
\item 3 de noviembre
\item 10 de noviembre
\item 22 de diciembre
\end{itemize}

\section{Plan de resolución de problemas}

Debido al caracter cambiante de los roles que toman los integrantes, no habrá
un recurso dedicado a la resolución de problemas. Para cada problema que surga,
será la persona más capaz en el área de experiencia del problema la encargada
de resolverlo.
