\documentclass[a4paper,11pt,oneside]{article}
\usepackage[pdftex]{graphicx}
\usepackage[utf8x]{inputenc}
\usepackage[spanish]{babel}
\usepackage{fancyhdr}
\usepackage{tabularx}
\usepackage{hyperref}
\usepackage{eurosym}
% \usepackage[usenames,dvipsnames]{color}
% \usepackage{colortbl}
% \usepackage[caption=false]{subfig}
% \usepackage{float}
% \usepackage{pdflscape}

\setlength{\headheight}{25pt}
\setlength{\parskip}{6pt}

% Margenes 1cm mas pequennos
\addtolength{\oddsidemargin}{-1cm}
\addtolength{\evensidemargin}{-1cm}
\addtolength{\textwidth}{2cm}
\addtolength{\voffset}{-1cm}
\addtolength{\textheight}{2cm}

\hypersetup{
colorlinks,
citecolor=black,
filecolor=black,
linkcolor=black,
urlcolor=black
}

\lhead{\includegraphics[height=20pt]{logo-umbrella.png}}
\chead{}
\rhead{Acta de reunión N° 1}

\begin{document}

\pagestyle{fancy}

%%%% Header %%%%%

\begin{center}
{\Large
\textbf{Acta de reunión del grupo de desarrollo software \textit{Umbrella
Corporation}.}
}
\end{center}
\textbf{Fecha:} 15 de septiembre de 2010\\
\textbf{Lugar:} Aula 0.06, Edificio Politécnico\\
\textbf{Asistentes:}\\
\hspace*{1cm}Ángel Durán Izquierdo\\
\hspace*{1cm}Antonio Gómez Poblete\\
\hspace*{1cm}Antonio Martín Menor de Santos\\
\hspace*{1cm}Daniel León Romero\\
\hspace*{1cm}Jorge Colao Adán\\
\hspace*{1cm}Laura Núñez Villa\\
\hspace*{1cm}Ricardo Ruedas García

%%%% end Header %%%%%

\section*{Puntos del día}

\begin{enumerate}
\item Creación del grupo
\item Análisis previo de requisitos
\end{enumerate}


\section*{Desarrollo de la reunión}

\subsection*{1. Creación del grupo}

Los asistentes a la reunión han acordado formar juntos un grupo de prácticas
para la asignatura de Ingeniería del Software II. No se ha propuesto ningún
otro miembro por lo que el grupo queda cerrado con los siete asistentes.

\subsection*{2. Análisis previo de requisitos}

Se ha realizado una lectura en profundidad del enunciado del problema y anotado
en papel un esbozo de los requisitos funcionales que tendrá la aplicación.
También se han anotado algunas cuestiones para resolverlas en las tutorías en
grupo.

\end{document}
