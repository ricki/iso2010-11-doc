\documentclass[a4paper,11pt,oneside]{article}
\usepackage[pdftex]{graphicx}
\usepackage[utf8x]{inputenc}
\usepackage[spanish]{babel}
\usepackage{fancyhdr}
\usepackage{tabularx}
\usepackage{hyperref}
\usepackage{eurosym}
% \usepackage[usenames,dvipsnames]{color}
% \usepackage{colortbl}
% \usepackage[caption=false]{subfig}
% \usepackage{float}
% \usepackage{pdflscape}

\setlength{\headheight}{25pt}
\setlength{\parskip}{6pt}

% Margenes 1cm mas pequennos
\addtolength{\oddsidemargin}{-1cm}
\addtolength{\evensidemargin}{-1cm}
\addtolength{\textwidth}{2cm}
\addtolength{\voffset}{-1cm}
\addtolength{\textheight}{2cm}

\hypersetup{
colorlinks,
citecolor=black,
filecolor=black,
linkcolor=black,
urlcolor=black
}

\lhead{\includegraphics[height=20pt]{logo-umbrella.png}}
\chead{}
\rhead{Acta de reunión N° 3}

\begin{document}

\pagestyle{fancy}

%%%% Header %%%%%

\begin{center}
{\Large
Acta de reunión del grupo de desarrollo software \textit{Umbrella
Corporation}.}
\end{center}
\textbf{Fecha:} 11 de octubre de 2010\\
\textbf{Lugar:} Aula de Libre Uso, Escuela de Informática\\
\textbf{Asistentes:}\\
\hspace*{1cm}Antonio Gómez Poblete\\
\hspace*{1cm}Daniel León Romero\\
\hspace*{1cm}Jorge Colao Adán\\
\hspace*{1cm}Laura Núñez Villa\\
\hspace*{1cm}Ricardo Ruedas García

%%%% end Header %%%%%

\section*{Puntos del día}

\begin{enumerate}
\item Asignación de tareas
\item Sesión de trabajo
\end{enumerate}


\section*{Desarrollo de la reunión}

\subsection*{1. Asignación de tareas}

Tras haber capturado previamente las funcionalidades de la aplicación, esta
semana trabajarán en su descripción los siguientes integrantes del grupo:
Antonio Gómez Poblete, Daniel León Romero, Jorge Colao Adán, y Laura Nuñez
Villa.

A Antonio Martín Menor de Santos se le ha encargado el prototipado de una
interfaz, haciendo especial énfasis en un mapa interactivo.

\subsection*{2. Sesión de trabajo}

Para definir algunas reglas de estilo y uniformidad en la redacción de la
especificación de requisitos, se ha trabajado en las tres primeras
funcionalidades de manera conjunta.

\end{document}
