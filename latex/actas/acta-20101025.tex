\documentclass[a4paper,11pt,oneside]{article}
\usepackage[pdftex]{graphicx}
\usepackage[utf8x]{inputenc}
\usepackage[spanish]{babel}
\usepackage{fancyhdr}
\usepackage{tabularx}
\usepackage{hyperref}
\usepackage{eurosym}
% \usepackage[usenames,dvipsnames]{color}
% \usepackage{colortbl}
% \usepackage[caption=false]{subfig}
% \usepackage{float}
% \usepackage{pdflscape}

\setlength{\headheight}{25pt}
\setlength{\parskip}{6pt}

% Margenes 1cm mas pequennos
\addtolength{\oddsidemargin}{-1cm}
\addtolength{\evensidemargin}{-1cm}
\addtolength{\textwidth}{2cm}
\addtolength{\voffset}{-1cm}
\addtolength{\textheight}{2cm}

\hypersetup{
colorlinks,
citecolor=black,
filecolor=black,
linkcolor=black,
urlcolor=black
}

\lhead{\includegraphics[height=20pt]{logo-umbrella.png}}
\chead{}
\rhead{Acta de reunión N° 5}

\begin{document}

\pagestyle{fancy}

%%%% Header %%%%%

\begin{center}
{\Large
Acta de reunión del grupo de desarrollo software \textit{Umbrella
Corporation}.}
\end{center}
\textbf{Fecha:} 25 de octubre de 2010\\
\textbf{Lugar:} Aula de Libre Uso, Escuela de Informática\\
\textbf{Asistentes:}\\
\hspace*{1cm}Ángel Durán Izquierdo\\
\hspace*{1cm}Antonio Gómez Poblete\\
\hspace*{1cm}Antonio Martín Menor de Santos\\
\hspace*{1cm}Daniel León Romero\\
\hspace*{1cm}Jorge Colao Adán\\
\hspace*{1cm}Laura Núñez Villa\\
\hspace*{1cm}Ricardo Ruedas García

%%%% end Header %%%%%

\section*{Puntos del día}

\begin{enumerate}
\item Resolución de problemas con Git
\item Asignación de tareas
\end{enumerate}


\section*{Desarrollo de la reunión}

\subsection*{1. Resolución de problemas con Git}

Algunos integrantes del grupo habían tenido problemas con Git durante la semana
anterior. Se ha realizado una sesión con sus propios portátiles en la que se
han configurado el entorno con sus datos y claves privadas. Se han realizado
todas las tareas habituales hasta que los miembros se han sentido seguros de
poder usarlo.

\subsection*{Asignación de tareas}

Con la iteración 4 queda cerrada la fase de inicio del proyecto. En esta
semana, Ángel Durán Izquierdo y Antonio Gómez Poblete iniciarán el trabajo de
programación implementando dos casos de uso.

Antonio Martín Menor de Santos, Daniel León Romero y Jorge Colao Adán
continuarán con el análisis y diseño de la aplicación, completando el trabajo
realizado en la semana anterior con cinco casos de uso más.

Laura Núñez Villa será la encargada de revisar y completar la especificación de
requisitos software.

\end{document}
