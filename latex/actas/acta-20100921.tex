\documentclass[a4paper,11pt,oneside]{article}
\usepackage[pdftex]{graphicx}
\usepackage[utf8x]{inputenc}
\usepackage[spanish]{babel}
\usepackage{fancyhdr}
\usepackage{tabularx}
\usepackage{hyperref}
\usepackage{eurosym}
% \usepackage[usenames,dvipsnames]{color}
% \usepackage{colortbl}
% \usepackage[caption=false]{subfig}
% \usepackage{float}
% \usepackage{pdflscape}

\setlength{\headheight}{25pt}
\setlength{\parskip}{6pt}

% Margenes 1cm mas pequennos
\addtolength{\oddsidemargin}{-1cm}
\addtolength{\evensidemargin}{-1cm}
\addtolength{\textwidth}{2cm}
\addtolength{\voffset}{-1cm}
\addtolength{\textheight}{2cm}

\hypersetup{
colorlinks,
citecolor=black,
filecolor=black,
linkcolor=black,
urlcolor=black
}

\lhead{\includegraphics[height=20pt]{logo-umbrella.png}}
\chead{}
\rhead{Acta de reunión N° 2}

\begin{document}

\pagestyle{fancy}

%%%% Header %%%%%

\begin{center}
{\Large
Acta de reunión del grupo de desarrollo software \textit{Umbrella
Corporation}.}
\end{center}
\textbf{Fecha:} 21 de septiembre\\
\textbf{Lugar:} Aula de Libre Uso, Escuela de Informática\\
\textbf{Asistentes:}\\
\hspace*{1cm}Ángel Durán Izquierdo\\
\hspace*{1cm}Antonio Gómez Poblete\\
\hspace*{1cm}Antonio Martín Menor de Santos\\
\hspace*{1cm}Daniel León Romero\\
\hspace*{1cm}Jorge Colao Adán\\
\hspace*{1cm}Laura Núñez Villa\\
\hspace*{1cm}Ricardo Ruedas García

%%%% end Header %%%%%

\section*{Puntos del día}

\begin{enumerate}
\item Nombre de la empresa
\item Herramientas de desarrollo
\end{enumerate}


\section*{Desarrollo de la reunión}

\subsection*{1. Nombre de la empresa}

Tras debatir algunos nombres, el grupo decidió con algunos votos en contra el
nombre de \textit{Umbrella Corporation}, basado en un conocido videojuego.

\subsection*{2. Herramientas de desarrollo}

Se ha debatido el uso de Eclipse o NetBeans como entorno de desarrollo, y al
final la mayoría de los desarrolladores se han decantado por Eclipse por la
variedad de \textit{plug-ins} relacionados con el proyecto.

Para el gestor de versiones, se quería elegir uno distribuido, y se eligió Git
porque algunos miembros ya tenían experiencia previa. Para el servicio de
\textit{hosting}, se ha elegido GitHub sobre Gitorious porque era más amigable
y disponía de traducción a español.

Para comunicarse, ya se había creado previamente un Wave, pero viendo que no
era eficaz, se ha decidido usar una lista de correo de Google Groups.

\end{document}
