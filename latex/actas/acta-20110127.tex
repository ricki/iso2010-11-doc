\documentclass[a4paper,11pt,oneside]{article}
\usepackage[pdftex]{graphicx}
\usepackage[utf8x]{inputenc}
\usepackage[spanish]{babel}
\usepackage{fancyhdr}
\usepackage{tabularx}
\usepackage{hyperref}
\usepackage{eurosym}
% \usepackage[usenames,dvipsnames]{color}
% \usepackage{colortbl}
% \usepackage[caption=false]{subfig}
% \usepackage{float}
% \usepackage{pdflscape}

\setlength{\headheight}{25pt}
\setlength{\parskip}{6pt}

% Margenes 1cm mas pequennos
\addtolength{\oddsidemargin}{-1cm}
\addtolength{\evensidemargin}{-1cm}
\addtolength{\textwidth}{2cm}
\addtolength{\voffset}{-1cm}
\addtolength{\textheight}{2cm}

\hypersetup{
colorlinks,
citecolor=black,
filecolor=black,
linkcolor=black,
urlcolor=black
}

\lhead{\includegraphics[height=20pt]{logo-umbrella.png}}
\chead{}
\rhead{Acta de reunión N° 6}

\begin{document}

\pagestyle{fancy}

%%%% Header %%%%%

\begin{center}
{\Large
Acta de reunión del grupo de desarrollo software \textit{Umbrella
Corporation}.}
\end{center}
\textbf{Fecha:} 27 de enero de 2011\\
\textbf{Lugar:} Aula de Libre Uso, Escuela de Informática\\
\textbf{Asistentes:}\\
\hspace*{1cm}Ángel Durán Izquierdo\\
\hspace*{1cm}Antonio Gómez Poblete\\
\hspace*{1cm}Jorge Colao Adán\\
\hspace*{1cm}Ricardo Ruedas García

%%%% end Header %%%%%

\section*{Puntos del día}

\begin{enumerate}
\item Plantilla para documentar las pruebas
\item Comprobación de los datos de entrada
\item Asignación de tareas
\end{enumerate}


\section*{Desarrollo de la reunión}

\subsection*{Definir plantilla para documentar las pruebas}

En la reunión se ha hablado sobre los elementos que los informes de pruebas
deberían contener. Se realizará un informe por cada caso de uso. En la cabecera
del informe se incluirán los siguientes elementos:

\begin{itemize}
\item Nombre del \textit{tester}
\item Fecha de asignación
\item Fecha de finalización
\item Partes del código bajo prueba
\end{itemize}

A continuación se detallarán las pruebas de desrrollo. Por cada operación
probada, se debe incluir la siguiente información:

\begin{itemize}
\item Lista de los valores de prueba para cada atributo, y criterio seguido.
\item Estrategia para obtener los casos de prueba (\textit{pairwise},
\textit{each choice}, casos interesantes, etc).
\item Tabla completa de los casos de prueba y el resultado esperado.
\end{itemize}

Tras ello se incluirá un guión de pruebas exploratorias por cada escenario
posible con los siguientes datos:

\begin{itemize}
\item Acciones previas
\item Acciones de prueba
\item Resultados esperados
\end{itemize}

Por último se incluirá una sección llamada ``\textit{feedback}'' en la que se
relatarán las acciones llevadas a cabo por el \textit{tester} y el programador.

\subsection*{Comprobación de los datos de entrada}

En la reunión se habló de la necesidad (o ausencia de ella) de comprobar la
corrección de los datos introducidos en las operaciones de las clases de
dominio. Aunque la capa de presentación pueda filtrar determinados atributos,
se decidió, para eliminar una dependecia innecesaria con la capa de
presentación, que todas las operaciones de dominio filtrarían los datos para
que cumplan con la lógica de dominio. A continuación se muestra una lista no
exhaustiva de las comprobaciones a realizar:

\begin{itemize}
\item Objetos nulos
\item Cadenas de texto vacías
\item Números fuera de rango
\end{itemize}

\subsection*{Asignación de tareas}

De manera inmediata, Antonio Gómez tiene asignada la tarea de probar el caso de
uso ``Registrarse'', y Ángel Durán la de modificar la implementación del caso
de uso ``Unirse a partida'', de acuerdo a las medidas adoptadas en esta reunión.

Al resto de miembros se les asignará otras tareas a lo largo de la semana
siguiente.

\end{document}
