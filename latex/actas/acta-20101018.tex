\documentclass[a4paper,11pt,oneside]{article}
\usepackage[pdftex]{graphicx}
\usepackage[utf8x]{inputenc}
\usepackage[spanish]{babel}
\usepackage{fancyhdr}
\usepackage{tabularx}
\usepackage{hyperref}
\usepackage{eurosym}
% \usepackage[usenames,dvipsnames]{color}
% \usepackage{colortbl}
% \usepackage[caption=false]{subfig}
% \usepackage{float}
% \usepackage{pdflscape}

\setlength{\headheight}{25pt}
\setlength{\parskip}{6pt}

% Margenes 1cm mas pequennos
\addtolength{\oddsidemargin}{-1cm}
\addtolength{\evensidemargin}{-1cm}
\addtolength{\textwidth}{2cm}
\addtolength{\voffset}{-1cm}
\addtolength{\textheight}{2cm}

\hypersetup{
colorlinks,
citecolor=black,
filecolor=black,
linkcolor=black,
urlcolor=black
}

\lhead{\includegraphics[height=20pt]{logo-umbrella.png}}
\chead{}
\rhead{Acta de reunión N° 4}

\begin{document}

\pagestyle{fancy}

%%%% Header %%%%%

\begin{center}
{\Large
Acta de reunión del grupo de desarrollo software \textit{Umbrella
Corporation}.}
\end{center}
\textbf{Fecha:} 18 de octubre de 2010\\
\textbf{Lugar:} Aula de Lbre Uso, Escuela de Informática\\
\textbf{Asistentes:}\\
\hspace*{1cm}Ángel Durán Izquierdo\\
\hspace*{1cm}Antonio Gómez Poblete\\
\hspace*{1cm}Antonio Martín Menor de Santos\\
\hspace*{1cm}Daniel León Romero\\
\hspace*{1cm}Jorge Colao Adán\\
\hspace*{1cm}Ricardo Ruedas García

%%%% end Header %%%%%

\section*{Puntos del día}

\begin{enumerate}
\item Asignación de tareas
\item Uso de Git
\end{enumerate}


\section*{Desarrollo de la reunión}

\subsection*{1. Asignación de tareas}

En esta semana se va a realizar la iteración 3 del proyecto. Antonio Gómez
Poblete y Jorge Colao Adán van a empezar a realizar el análisis y diseño de los
tres primeros casos de uso.

Los otros cuatro integrantes del grupo terminarán de especificar los requisitos
funcionales de la aplicación, trabajando en grupos de dos.

Se ha enviado un mensaje a la lista de correo con información más detallada
sobre las tareas asignadas a cada grupo.

\subsection*{Uso de Git}

A continuación se han explicado los comandos básicos para utilizar el sistema
de control de versiones Git, haciendo uso de ramas y repositorios remotos.

\end{document}
