\documentclass[a4paper,11pt,oneside]{article}
\usepackage[pdftex]{graphicx}
\usepackage[utf8x]{inputenc}
\usepackage[spanish]{babel}
\usepackage{fancyhdr}
\usepackage{tabularx}
\usepackage{hyperref}
\usepackage{eurosym}
% \usepackage[usenames,dvipsnames]{color}
% \usepackage{colortbl}
% \usepackage[caption=false]{subfig}
% \usepackage{float}
% \usepackage{pdflscape}

\setlength{\headheight}{25pt}
\setlength{\parskip}{6pt}

% Margenes 1cm mas pequennos
\addtolength{\oddsidemargin}{-1cm}
\addtolength{\evensidemargin}{-1cm}
\addtolength{\textwidth}{2cm}
\addtolength{\voffset}{-1cm}
\addtolength{\textheight}{2cm}

\hypersetup{
colorlinks,
citecolor=black,
filecolor=black,
linkcolor=black,
urlcolor=black
}

\lhead{\includegraphics[height=20pt]{logo-umbrella.png}}
\chead{}
\rhead{Acta de reunión N° 7}

\begin{document}

\pagestyle{fancy}

%%%% Header %%%%%

\begin{center}
{\Large
Acta de reunión del grupo de desarrollo software \textit{Umbrella
Corporation}.}
\end{center}
\textbf{Fecha:} 14 de febrero de 2011 \\
\textbf{Lugar:} Aural de Libre Uso, Escuela Superior de Informática \\
\textbf{Asistentes:}\\
\hspace*{1cm}Ángel Durán Izquierdo\\
\hspace*{1cm}Antonio Gómez Poblete\\
\hspace*{1cm}Antonio Martín Menor de Santos\\
\hspace*{1cm}Daniel León Romero\\
\hspace*{1cm}Jorge Colao Adán\\
\hspace*{1cm}Laura Núñez Villa\\
\hspace*{1cm}Ricardo Ruedas García

%%%% end Header %%%%%

\section*{Puntos del día}

\begin{enumerate}
\item Nueva organización de trabajo para la implementación
\end{enumerate}


\section*{Desarrollo de la reunión}

\subsection*{Nueva organización de trabajo para la implementación}

Dado el escaso tiempo que queda, y teniendo en cuenta que hay varios casos de
uso diseñados y esperando a ser implementados, se asignaron en paralelo a
varios programadores, pero ésto provocó algunas dependencias en código que no
habrían ocurrido de haber sido implementados de manera secuencial.

Debido a ésto, los programadores trabajarán sobre áreas del código
independientes y aportarán a todos los casos de uso pendientes de implementar.

La asignación de tareas queda como sigue:

\begin{itemize}
\item Ángel Durán Izquierdo: Está encargado de finalizar la capa de
comunicaciones. Además completará las operaciones que falten de implementar en
el gestor de usuarios. Las clases sobre las que trabajará serán:
\subitem \texttt{ServerAdapter}
\subitem \texttt{ClientAdapter}
\subitem \texttt{ClientCallback}
\subitem \texttt{UserManager}

\item Antonio Gómez Poblete: Declarará todas las operaciones de las clases de
gestión de partidas y motor de juego y adaptará las operaciones existentes a
la nueva clase \texttt{TerritoryDecorator}. También creará la nueva clase
para los ataques. Las clases sobre las que trabajará serán:
\subitem \texttt{GameManager}
\subitem \texttt{GameEngine}
\subitem \texttt{ClientCallback}
\subitem \texttt{Attack}

\item Antonio Martín Menor de Santos: Realizará los diálogos de referentes a los
casos de uso actuales. Las clases sobre las que trabajará serán:
\subitem \texttt{LaunchAttackDialog}
\subitem \texttt{ReplyAttackDialog}
\subitem \texttt{MoveUnitsDialog}
\subitem \texttt{BuyUnitsDialog}

\item Daniel León Romero y Jorge Colao Adán: Trabajarán juntos en corregir el
funcionamiento de la clase \texttt{MapView} y revisar los modelos de datos. Las
clases sobre las que trabajará serán:
\subitem \texttt{MapView}
\subitem \texttt{PlayerListModel}
\subitem \texttt{MapModel}

\item Laura Núñez Villa: Creará la nueva clase \texttt{TerritoryDecorator} y
seguirá trabajando en el análisis y diseño de los siguientes casos de uso.
\end{itemize}


\end{document}
