\section{Propósito}

Este documento trata de describir detalladamente las funcionalidades que el
cliente para el juego \textit{La Conquista del Mundo} debe tener. Estas
descripciones servirán a los analistas y diseñadores del sistema para
desempeñar su trabajo en siguientes iteraciones del desarrollo.

\section{Alcance}

El producto final del desarrollo será una aplicación cliente en Java para el
juego \textit{La Conquista del Mundo}. Esta aplicación permite a un usuario
nuevo registrarse en el sistema, y una vez registrado, jugar una partida con
otros usuarios. La aplicación deberá mantener una copia de los datos de la
partida y comunicarse con el servidor cada vez que el usuario realice cualquier
acción, y recibir las acciones del resto de jugadores.

\section{Definciones, acrónimos y abreviaturas}

\begin{description}
\item [Continente] Conjunto de regiones agrupadas en una determinada zona del
mapa.
\item [Middleware] Librería que conecta aplicaciones separadas por red.
\item [Región] Parta indivisible del tablero de juego.
\item [RMI] \textit{Middleware} de Java.
\item [Tablero] Mapa sobre el que se juega una partida.
\end{description}

\section{Referencias}

Para la elaboración de este documento se ha seguido ampliamente el siguiente
estándar:\\
\\
IEEE Std 830-1998, \textit{Recommended Practice for Software Requirements
Specifications}.
The Institute of Electrical and Electronics Engineers, Inc. 1998.\\

Además, se han tomado como ejemplo a seguir los siguientes documentos:\\
\\
{\small
\url{http://www.cse.msu.edu/~chengb/RE-491/Papers/SRSExample-webapp.doc}\\
\url{http://mcis.jsu.edu/studio/SRSSample.doc}\\
}\\

\section{Visión general del documento}

El siguiente capítulo, Descripción general, da una idea general de las funciones
que debe tener el producto y sirve para establecer un contexto para la
especificación técnica de los requisitos que viene en el capítulo 3, Requisitos
específicos.

Este tercer capítulo está escrito principalmente para los desarrolladores de la
aplicación y describe en términos técnicos los detalles de cada funcionalidad.
